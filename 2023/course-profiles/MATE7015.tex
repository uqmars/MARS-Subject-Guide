\hypertarget{MATE7015}{\section{MATE7015 - Additive Manufacturing}}

\large
\textcolor{turbo_purple}{\href{https://my.uq.edu.au/programs-courses/course.html?course_code=MATE7015}{Official Page}} \\
Rating: \cstar\cstar\cstar\cstar\ostar

\normalsize
\subsection*{Description}
Additive Manufacturing (AM), also known as 3D printing, is growing at a rapid rate with global manufacturers increasingly realising the benefits of producing parts by AM.
In 2017 there was 80\% year on year growth in metal AM system sales, and over an 800\% increase from 2012 [1].
AM is becoming a mainstream manufacturing process because it can not only reduce manufacturing time and costs but also offers flexibility to produce geometrically complex designs with superior functionality.
An example is newly optimised 3D printed metal aircraft brackets that are 50\% lighter, use 90\% less material and 90\% less energy to produce compared to the equivalent bracket currently produced by machining.
Given that AM will continue to grow, future engineers must adapt to this revolutionary manufacturing process so now is a timely opportunity to introduce a dedicated course that prepares our graduates for working with this technology.

\subsection*{Review}
review here
