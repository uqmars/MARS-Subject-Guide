\chapter{Course Profiles}

% The following is the course listing layout that can be used as a template:
% \course{Course Code}
%	{Course Name}
%	{Course rating out of 5}
%	{Prerequisites}
%	{Recommended Prerequisites}
%	{Companion Courses}
%	{Course Description}
%	{Course Review}

\course{AERO4300}
	{Aerospace Composities}
	{4}
	{MECH2300}
	{}
	{}
	{Application of composite materials used in the aerospace industry. Characteristics of composite materials; Analysis of the stiffness & strength behaviour of fibre-reinforced composite & sandwich structures; Composite materials manufacturing processes & techniques used in the aerospace industry. The course provides an introduction to composite materials, manufacturing and design with a primary focus on applications in the aerospace industry. Students will gain insight into the commonly used constituent materials, intermediates, and resulting composites and understand the many choices available in this hierarchy. The course will then introduce the students to the most common composite manufacturing processes for both thermoplastic and thermosetting matrix composites. Emphasis is put on enabling students to select the most appropriate manufacturing process for a given application and material combination. The design component of the course covers the key concepts of structural design and evaluation of composite materials. This includes introducing students to micromechanics, classical-laminate theory, and failure theory of composite materials using both analytical as well as numerical methods. In the final two weeks of the semester, the course ventures into advanced topics, including sandwich structures, degradation and durability, and ceramic matrix composite materials and manufacturing. The learning experience is enhanced with two hands-on manufacturing practicals and a non-destructive testing workshop.}
	{review here}

\course{AERO4450}
	{Aerospace Propulsion}
	{4}
	{MECH2700, MECH3400, MECH3410}
	{}
	{}
	{Air-breathing propulsion systems; rocket propulsion systems; combustion applied to aerospace propulsion systems.}
	{review here}

\course{AERO4470}
	{Hypersonics}
	{4}
	{MECH3410}
	{}
	{}
	{Hypersonic gas dynamics including:  hypersonic wind tunnels, flow deflection techniques, viscous flows, non-equilibrium flows, rarefied gas dynamics, and satellite drag.}
	{review here}

\course{AERO4800}
	{Space Engineering}
	{4}
	{MECH2210, MECH3400, MECH2700}
	{}
	{}
	{This course draws upon and extends many of the methods used by mechanical and space engineers in their professional practice. In this course you will learn how to perform advanced trajectory design and launch vehicle sizing; analyse issues related to thermal loading, power, and control of space systems; analyse how spacecraft safely return to Earth; perform reliability estimates for complex systems; and communicate your ideas and concepts through in-class exercises and written reports.}
	{review here}

\course{COMP3506}
	{Algorithms and Data Structures}
	{4}
	{CSSE2002 and (MATH1061 or (CSSE2010 and STAT2202))(STAT2202 no longer offered)}
	{}
	{}
	{Data structures and types, mapping of abstract information structures into representations on primary and secondary storage. Analysis of time and space complexity of algorithms. Sequences. Lists. Stacks. Queues. Sets, multisets, tables. Trees. Sorting. Hash tables. Priority queues. Graphs. String algorithms.}
	{review here}

\course{COMP3702}
	{Artificial Intelligence}
	{4}
	{(CSSE1001 or CSSE7030) or ENGG1001}
	{}
	{}
	{Methods and techniques within the field of artificial intelligence, including topics on search, reasoning and planning with certainty, decision-making under uncertainty, learning to act and reasoning about other agents. Specific emphasis on the practical utility of algorithms and their implementation in software.}
	{review here}

\course{COMP3710}
	{Pattern Recognition and Analysis}
	{4}
	{(MATH1051 or MATH1071) and (CSSE1001 or ENGG1001)}
	{}
	{}
	{Understanding patterns in our environment is an important cognitive ability. The development of recognition and automated algorithms that are able to process copious amounts data without (or with limited) human intervention is critical in replicating this ability in machines. This course will cover the fundamentals of creating computational algorithms and models that are able to recognise and/or analyse patterns within data of various forms. Topics and algorithms will include fractal geometry, classification methods such as random forests, recognition approaches using deep learning and models of the human vision system. Python, the open-source paradigm and state-of-the-art packages like TensorFlow, JAX and PyTorch will be used as a mechanism for students to study patterns in nature and solve problems from various real-worlds data sources, such images, social media and biomedical signals.}
	{review here}

\course{COMP4702}
	{Machine Learning}
	{4}
	{(CSSE1001 or ENGG1001) and (MATH1051 or MATH1071) and (STAT1201 or STAT2003 or STAT2201 or STAT2202 or STAT2203)(STAT2202 no longer offered)}
	{}
	{}
	{Machine learning is a branch of artificial intelligence concerned with the development & application of adaptive algorithms that use example data or previous experience to solve a given problem. Topics include learning problems (e.g. regression, classification, unsupervised, reinforcement) & theory, neural networks, statistical & probabilistic models, clustering, ensembles, implementation issues, applications (e.g. bioinformatics, cognitive science, forecasting, robotics, signal & image processing).}
	{review here}

\course{COSC2500}
	{Numerical Methods in Computational Science}
	{4}
	{SCIE1000 or (CSSE1001 and MATH1051 or MATH1071) or (MATH1051 or MATH1071 and MATH1052 or MATH1072)}
	{}
	{}
	{This course provides an introduction to basic numerical methods and computer programming for the solution of a number of classes of scientific problems. The course is interdisciplinary in nature, incorporating a number of case studies in biology, physics, chemistry, and engineering.}
	{review here}

\course{COSC3000}
	{Vizualization, Computer Graphics & Data Analysis}
	{4}
	{COSC2500 or CSSE2002}
	{}
	{}
	{Scientific visualisation is the use of images to provide insight into phenomena, and is a key tool for the analysis and understanding of biological, physical and engineering processes. It is becoming ever more important as the size of data sets continue to grow due to increasing power of computers and measurement devices. This course provides an introduction to computer graphics, data analysis and visualisation as tools to understand and interpret real world data and output from large-scale computational models.}
	{review here}

\course{COSC3500}
	{High-Performance Computing}
	{4}
	{COSC2500 or CSSE2002 or CSSE2010 or CSSE2310 or PHYS3071}
	{}
	{}
	{This course teaches the methods and technology of high-performance computing. Primarily a project-based course, students select a scientific model they are interested in parallelising and then apply some or all of the techniques of vectorisation (SIMD, AVX), threading (MIMD, openMP), GPU programming (CUDA), and/or cluster computing (MPI) to accelerate their computation.}
	{review here}

\course{CSSE1001}
	{Introduction to Software Engineering}
	{4}
	{None}
	{}
	{}
	{Introduction to Software Engineering through programming with particular focus on the fundamentals of computing and programming, using an exploratory problem-based approach. Building abstractions with procedures, data and objects; data modelling; designing, coding and debugging programs of increasing complexity}
	{review here}

\course{CSSE2002}
	{Programming in the Large}
	{4}
	{CSSE1001 or ENGG1001}
	{}
	{}
	{Working on large and complex software systems and ensuring those systems remain maintainable requires disciplined, individual practices. Software must be well-specified, well-implemented and well-tested. This course covers concepts and techniques in modern programming languages that help support good practice (such as OO concepts, genericity and exception handling) with specific application to file IO and GUIs in Java.}
	{review here}

\course{CSSE2010}
	{Introduction to Computer Systems}
	{4}
	{CSSE1001 or ENGG1001}
	{}
	{}
	{Introduction to digital logic and digital systems; machine level representation of data; computer organization; memory system organization and architecture; interfacing and communication; microcontroller architecture and usage; programming of microcontroller-based systems.}
	{review here}

\course{CSSE2310}
	{Computer Systems Principles and Programming}
	{4}
	{CSSE1001 or ENGG1001}
	{}
	{}
	{Systems Programming in C. Operating Systems Principles: memory management, basics of machine organization, file systems, processes and threads, interprocess communication. Computer Networks Principles: topologies and models of computer networks, protocols, network programming, network applications.}
	{review here}

\course{CSSE3010}
	{Embedded Systems Design and Interfacing}
	{4}
	{CSSE2010 and CSSE2310}
	{}
	{}
	{Microcontroller system hardware and software. C programming for embedded microcontroller and peripheral devices. Principles and practice of using Embedded RTOS (Real Time Operating System) and peripheral devices such as sensors and actuators to build a small embedded system. Peripheral interfacing methods and standards. Analog-digital conversion methods and interfacing. Basics of digital communication signals, modulation schemes and error correction methods. Data compression, formats for audio, image and video coding.}
	{review here}

\course{CSSE4010}
	{Digital System Design}
	{4}
	{CSSE1000 (currently not offered) or CSSE2010}
	{}
	{}
	{The objective of this course is to give the students the theoretical basis and practical skills in modern design of medium size digital systems in various technologies, with a focus on Field Programmable Gate Arrays (FPGAs). The design methodology, systematically introduced and used in the course, is based on simulation and synthesis with hardware description language (VHDL) tools. Topics covered in this course include: conceptual design step from requirements and specification to simulation and synthesis model in VHDL, design of complex controllers with Finite State Machines, design of sequential blocks with Controller-Datapath methodology, issues in design for testability, electrical and timing issues in logic and system design, overview of implementation technologies with emphasis on advances in FPGAs.}
	{review here}

\course{CSSE4011}
	{Advanced Embedded Systems}
	{4}
	{CSSE3010}
	{}
	{}
	{Advanced topics in Embedded System, including wireless networks and wireless sensor networks.}
	{review here}

\course{CSSE7610}
	{Concurrency: Theory and Practice}
	{4}
	{CSSE2002 or CSSE7023}
	{}
	{}
	{Provides a solid understanding of the issues of concurrent programming - processes and threads, scheduling, synchronisation, communications, and data sharing - including their application in distributed systems. The course covers methods for both the specification and verification of such systems at a high level of abstraction, and their implementation in a modern programming language.}
	{review here}

\course{CSSE7023}
	{Advanced Software Engineering}
	{4}
	{CSSE7030}
	{}
	{}
	{Working on large and complex software systems and ensuring those systems remain maintainable requires disciplined, individual practices. Software must be well-specified, well-implemented and well-tested. This course covers concepts and techniques in modern programming languages that help support good practice (such as OO concepts, genericity and exception handling) with specific application to file IO and GUIs in Java.}
	{review here}

\course{DATA2001}
	{Fundamentals of Data Science}
	{4}
	{(CSSE1001 or ENGG1001) and INFS1200}
	{}
	{}
	{This course will utilize a scenario-based methodology to approach simple and complex data science problems in various data-intensive sectors and domains.}
	{review here}

\course{DSGN1100}
	{Design: Interaction}
	{4}
	{None}
	{}
	{}
	{DSGN1100 introduces foundational visual communication within the design process. Students learn a variety of design techniques, from sketching methods to digital visualisation, to effectively convey design concepts. The course explores essential design principles and processes, enabling students to craft effective communication for the design solutions.}
	{review here}

\course{DSGN1200}
	{Design: Experience}
	{4}
	{DSGN1100 or DSGN1500}
	{}
	{}
	{This course explores the significance of design in shaping user experience. It focuses on the capacity of design to engage human behaviours that can imbue forms and spaces with new cultural meanings. Spatial and programmatic studies are undertaken of contemporary settings where design can be used to activate experience. Learning is undertaken in a collaborative studio setting where students will develop a sensibility for the visual and spatial in design.}
	{review here}

\course{DSGN1500}
	{Design for a Better World}
	{4}
	{None}
	{}
	{}
	{This course introduces design as a discipline including its historical context, the design process, and design methods. Designing aims to improve situations. It typically begins with a problem or challenge. This problem is influenced by various social, economic, cultural, political, and environmental factors. As designers grapple with the problem, they consider different ways to solve it using a variety of methods. The design process is iterative and enhances designers’¿ capacity for proposing relevant solutions. It involves understanding contextual factors, defining the problem, and determining the means of finding solutions new challenges through imagination and judgment. Through practice and a thorough understanding of history and context, designers, in collaboration with diverse stakeholders, become capable of addressing complex issues and creating positive change.}
	{review here}

\course{DSGN2100}
	{Design: Organisation}
	{4}
	{4 DSGN units at level 1}
	{}
	{}
	{This course focuses on design thinking as a facilitator of social and cultural organisation in public settings. It looks particularly at incorporating intercultural awareness into the arrangement and understanding of places reflecting diverse cultures within Australia and globally. Students will understand how people take cues from their surroundings and deploy that knowledge in response to a set design project. Learning is undertaken in a collaborative studio setting where students will develop a sensibility for strategic design thinking.}
	{review here}

\course{DSGN2200}
	{Design: Environment}
	{4}
	{4 DSGN units at level 1}
	{}
	{}
	{This course will address concepts of sustainability and resilience that can inform design in response to environmental change. The ethical dimension of design thinking will be emphasised in tackling issues and problems that have wide social and economic impact. Learning is undertaken in a collaborative studio setting where students will develop a sensibility for strategic design thinking and critical discussion.}
	{review here}

\course{DSGN3100}
	{Design: Environment}
	{4}
	{6DSGN units at level 1 and 2 DSGN units at level 2}
	{}
	{}
	{This course addresses the important role of design in the making of public infrastructure. Emphasis is on developing innovative opportunities for design to effect outcomes in relating people with public infrastructure projects. Learning is undertaken in a collaborative studio setting focused on advancing leadership and team building skills relevant to engaging multi-disciplinary working groups and community stakeholders involved in the delivery of public infrastructure.}
	{review here}

\course{ELEC2004}
	{Circuits, Signals and Systems}
	{4}
	{ENGG1300}
	{}
	{}
	{Mathematical models of electrical components, circuits and systems. Time and frequency response. Issues in building complex systems from subsystems, including feedback. Signal theory and filter design. Theoretical investigations, substantial case studies and laboratory experiments.}
	{review here}

\course{ELEC2300}
	{Fundamentals of Electromagnetism and Electromechanics}
	{4}
	{ENGG1300 and (MATH1051 or MATH1071) and (MATH1052 or MATH1072) and (PHYS1171 or High School Physics)}
	{}
	{}
	{This course covers fundamental principles of electromagnetism, rotating electrical machines and power transformers. The course is intended to link underlying physics of electromagnetic fields to the operation of electrical machines. The learning activities include substantial case studies & laboratory experiments.}
	{review here}

\course{ELEC2400}
	{Electronic Devices and Circuits}
	{4}
	{ENGG1300 and (MATH1051 or MATH1071) and (MATH1052 or MATH1072)}
	{}
	{}
	{Physical models of semiconductor devices. Analysis and design of common electronic circuits using discrete semiconductor devices and operational amplifiers. Examples of use in analysis & design of amplifiers, analogue signal conditioning, filters and other circuits.}
	{review here}

\course{ELEC3004}
	{Signals, Systems and Control}
	{4}
	{ELEC2004 and (STAT2201 or STAT2202) and (MATH2000 or MATH2001)(STAT2202 NO LONGER OFFERED and MATH2000 has been replaced by MATH2001)}
	{}
	{}
	{Discrete-time signals and systems, system properties (linearity, time-invariance, memory, causality, stability), sampling and reconstruction, A/D and D/A converters, DFT/FFT, z transform, stochastic processes, frequency-selective filters, effect of feedback, introduction to control.}
	{review here}

\course{ELEC3100}
	{Fundamentals of Electromagnetic Fields and Waves}
	{4}
	{(ELEC2003 or ELEC2300) and (MATH2000 or MATH2001) (ELEC2003 has been replaced by ELEC2300)}
	{}
	{}
	{Fundamentals of engineering electromagnetics including transmission lines, time varying fields, plane waves, waveguides, radiation and basic antennas. Applications in area of communications and sensors.}
	{review here}

\course{ELEC3310}
	{Electrical Energy Conversion and Utilisation}
	{4}
	{ELEC2003 or ELEC2300 (ELEC2003 has been discontinued)}
	{}
	{}
	{Generation of electricity. Three phase balanced circuits; magnetic circuits. Transformers. Harmonics. Steady state analysis of dc. Synchronous and induction machines. Special motors. Modern motor control systems. Demand side management. Renewable energy sources. Distributed generation and uninterruptible power supplies.}
	{review here}

\course{ELEC4310}
	{Power Systems Analysis}
	{4}
	{ELEC3300 or ELEC3310(ELEC330 is replaced by ELEC3310)}
	{}
	{}
	{Overview of power system modelling, load flow analysis, symmetrical & unsymmetrical fault calculation, power system stability, application of software tool for power system analysis, distribution network and voltage regulation, basic market structure.}
	{review here}

\course{ELEC4410}
	{Advanced Electronic and Power Electronics Design}
	{4}
	{ELEC2400 or ELEC3400(ELEC3400 replaced by ELEC2400)}
	{}
	{}
	{This course is about power processing using control, signal processing and electronic systems. The application can be from many megawatts in an electric train or a wind farm to less than a few watts in a digital camera or a mobile phone. For modern power electronics system, computer simulation tools are used to analyse and optimise a design before any prototyping. Topics covered in this course are a) power electronic systems and applications such as motor drive, renewable energy and power supply systems; b) energy conversion topologies; c) modelling and control; d) design factors and computer simulations.}
	{review here}

\course{ELEC4620}
	{Digital Signal Processing}
	{4}
	{ELEC3004}
	{}
	{}
	{Advanced digital filtering: polyphase, multirate, all-pass, lattice and IIR filters. Signal conditioning, analogue filter types, sigma delta converters. Fast algorithms; Cooley-Tukey FFT, mixed radix formulations, Good-Thomas algorithm. Autoregressive, moving average signals. DSP applications and programming.}
	{review here}

\course{ELEC4630}
	{Computer Vision and Deep Learning}
	{4}
	{ELEC3004}
	{}
	{}
	{Image sensors, colour models, discrete cosine transform, image & video compression. Computer vision, morphological techniques, watershed transform, skeletonisation, image segmentation, active contours.}
	{review here}

\course{ENGG1100}
	{Professional Engineering}
	{4}
	{None}
	{}
	{}
	{Introduction to engineering design through a discipline-specific team project. Students will learn and apply professional engineering concepts and issues including: sustainability, safety, estimation, materials selection, decision making, project management, information literacy, communication (graphics, written, oral), ethics, and prototyping (building). The course provides an introduction to engineering as a profession.}
	{review here}

\course{ENGG1300}
	{Introduction to Electrical Systems}
	{4}
	{Mathematical Methods, Maths B or MATH1040}
	{}
	{}
	{Introduction to electrical circuits and systems. Solution of simple AC and DC Circuits. Electrical units and measurements. Voltage, current, impedance. Equivalent circuits. Electrical energy and power. Resistors, inductors, capacitors, phasors, filters. Introduction to analog and digital telecommunication systems. Operational amplifiers, sensors and actuators, simple controllers. Use of laboratory instruments, simulators and mathematical software tools.}
	{review here}

\course{ENGG1700}
	{Statics and Materials}
	{4}
	{Mathematical Methods, Maths B or MATH1040}
	{}
	{}
	{In this course students will: - Develop conceptual understanding and fundamental engineering skills in statics and materials across a range of contexts. - Combine their knowledge of statics and materials to an authentic design task. - Engage in hands on learning, including design, build, and test activities}
	{review here}

\course{ENGG4103}
	{Engineering Asset Management}
	{4}
	{None}
	{}
	{}
	{Fundamentals of Reliability Engineering: failure intensity functions; system reliability; exploratory data analysis; the Weibull function; design for reliability; HAZOP; FMECA. Maintenance Management; preventive, predictive, proactive and corrective methods and their place in maintenance strategy; maintenance performance indices; workforce estimation and organisational structure; spare parts administration; maintenance contracts and contract administration; reliability centred maintenance; total productive maintenance. Preventive component replacement and capital equipment replacement decisions.}
	{review here}

\course{ENGG4900 (NO LONGER OFFERED)}
	{Professional Practice and the Business Environment}
	{4}
	{See description for more details}
	{}
	{}
	{ This course has been replaced by ENGG4901 (semester 1) and ENGG4902 (semester 2). Students should complete one of ENGG4901 or ENGG4902 in place of ENGG4900. Professional Practice is designed to give you the knowledge needed to effect change and implement design solutions in the real world. You will be able to identify barriers to technology uptake and work towards overcoming these through practical knowledge of: engineering economics, engineering law, engineering ethics, and the nature of engineering businesses. Students will learn how to undertake and interpret cost-benefit analyses, develop the skills required to understand business decision-making and economic drivers relevant to engineering and investigate key concepts required for ethical professional practice. Industry representatives and academics will deliver keynote lectures. Students will engage in workshops and project-based discipline-specific content and assessment which will lead the student through the issues encountered in professional engineering practice. Assessment will have both written and oral sections.}
	{review here}

\course{ENGG7291}
	{Engineering Placement A}
	{4}
	{None}
	{}
	{}
	{A major investigation or research project or a significant design task in industry, UQ or another university/research institute that integrates factors encountered in real life industry or research projects. Please note: This course falls outside the regular semester dates. Students are required to undertake a placement of up to 24 weeks in duration between January to June for semester 1. This course is restricted to BE(Hons)/ME students in the Fields of Electrical, Electrical & Computer, Electrical & Biomedical, Mechanical, Mechanical and Aerospace, Mechatronic, Software Engineering and and ME Urban Water Engineering (Professional) students. Students need to contact EAIT Employability (employability@eait.uq.edu.au) to make arrangements for suitable placements in the semester prior to their required placement semester.}
	{review here}

\course{ENGG7302}
	{Advanced Computational Techniques in Engineering}
	{4}
	{MATH2001 and MATH2010 and (STAT2201 or STAT2202)(STAT2202 no longer offered)}
	{}
	{}
	{An advanced course designed to deepen student knowledge and capability in computational techniques in areas of particular importance to engineering. Topics are drawn from linear algebra, stochastic systems and optimisation theory with emphasis on applications and examples in various fields of engineering including but not limited to biomedical engineering, electricity market, embedded systems and microwave and telecommunications. Practical skills in MATLAB programming are developed.}
	{review here}

\course{ENGG7701(NOT OFFERED)}
	{Engineering Grand Challenges}
	{4}
	{}
	{}
	{}
	{This course is not offered in 2024. Implications of being a professional engineer in the 21st century. Human forces: socio-political, psychology, economics and leadership. Societal Grand Challenges. Improved communications skills.}
	{review here}

\course{ENGG7811}
	{Research Methods}
	{4}
	{None}
	{}
	{}
	{ENGG7811 Research Methods will be discontinued and offered as REIT6811 Research Methods from Semester 1, 2025¿}
	{review here}

\course{ENGY4000}
	{Energy Systems}
	{4}
	{(CHEE3020 and CHEE3004) or MECH3400}
	{}
	{}
	{This course provides an overview of a wide range of energy systems including energy production from renewable (solar, wind, hydro, ocean, biomass) and non-renewable (fossil and nuclear) resources in the context of climate change and energy transitions. A range of engineering principles will be consolidated (i.e. mass & energy balances, thermodynamic cycles, process optimisation & power generation) in tandem with the application of sustainable development principles and business perspectives.}
	{review here}

\course{FIRE3700}
	{Introduction to Fire Safety Engineering}
	{4}
	{CIVL2131 and CIVL2330}
	{}
	{}
	{This course provides an introduction to the implementation of fire safety in infrastructure, industry and vehicles. The focus of the course is to establish the knowledge and rationale followed when bringing safety into the design process.}
	{review here}

\course{INFS1200}
	{Introduction to Information Systems}
	{4}
	{None}
	{}
	{}
	{Information systems analysis, design and implementation, relational database technology, data modelling, data querying using SQL, building a small scale information systems using a relational database management system.}
	{review here}

\course{INFS2200}
	{Relational Database Systems}
	{4}
	{INFS1200}
	{}
	{}
	{Concepts needed to build large information management systems using current technology; performance management of relational data, storage and access, index structures, views and query processing, transaction management, concurrency control and recovery.}
	{review here}

\course{INFS3208}
	{Cloud Computing}
	{4}
	{CSSE1001 and INFS1200}
	{}
	{}
	{As a major computing infrastructure, Cloud Computing provides the modern on-demand services for management and usage of large and shared computing resources including storage, computations and communications. This course will cover in-depth knowledge for Cloud Computing and the practical experience in designing and implementing large-scale and composite business web applications on Cloud Computing platform. This course covers a wide range of cloud computing-related X-as-a-Service technologies, including Software-as-a-Service (SaaS), Platform-as-a-Service (PaaS), Infrastructure-as-a-Service (IaaS), Data-as-a-Service (DaaS), and related technologies such as Cloud Computing Ecosystem. For delivering scalable computing services in a pay-as-you-go model via the Internet, Cloud Computing approaches are used to deal with effective and efficient development and deployment problems of web services and information systems with particular focus on big data challenges that arise across a variety of domains.}
	{review here}

\course{INFS4203}
	{Data Mining}
	{4}
	{(CSSE1001 or ENGG1001) and INFS2200}
	{}
	{}
	{Techniques used for data cleaning, finding patterns in structured, text and web data; with application to areas such as customer relationship management, fraud detection and homeland security.}
	{review here}

\course{MATE4302}
	{Electrochemistry and Corrosion}
	{4}
	{None}
	{}
	{}
	{Fundamental of electrochemical reactions, thermodynamics and kinetics of electrochemical reactions, mass transfer/diffusion in electrolytes, electrochemical method of analysis, applications (fuel cells, re-chargeable batteries, super-capacitors, and photo-electrochemical reactions), corrosion fundamentals, design against corrosion, corrosion protection principles and practice, corrosion in common environments, corrosion resistant alloys.}
	{review here}

\course{MATE7013}
	{Advanced Manufacturing}
	{4}
	{None}
	{}
	{}
	{Current global problems require increasingly sophisticated materials and appropriate advanced approaches and methodologies for their manufacture. This course will look at materials design for device manufacture, manufacturing techniques and manufacturing systems that are used to deliver innovative products and devices from the laboratory to industrial production. Several key manufacturing techniques, such as nano-, electronic and sustainable manufacturing will be covered as case studies illuminating how materials and manufacturing processes affect the end performance of the product, the economics of production and the impact on society and the environment. To obtain greater insight into smart manufacturing processes, students will complete projects, literature reviews/lab reports and oral/poster presentations in specific areas of manufacturing.}
	{review here}

\course{MATE7014}
	{Advanced Materials Characterization}
	{4}
	{None}
	{}
	{}
	{Materials Characterization provides unique tools for understanding the materials and their demonstrated properties. Materials Characterization techniques, such as x-ray diffraction, scanning electron microscopy, and transmission electron microscopy, allow detailed structural, chemical, and morphological characteristics of materials to be determined, which has become essential tools for materials research and their productions. By corelating the determined structural and chemical characteristics of a material with its fabrication/processing, the formation mechanism of the material can be clarified. This is vital for developing new material systems, and for identifying problems in the production lines. On the other hand, the correlation of the determined structural and chemical characteristics of a material with its demonstrated properties allows the material's structure-property link to be built, which is critically important for understanding the origin of the properties. For this reason, demand for learning various materials characterization techniques have increased sharply in the recent decades.}
	{review here}

\course{MATE7015}
	{Additive Manufacturing}
	{4}
	{None}
	{}
	{}
	{Additive Manufacturing (AM), also known as 3D printing, is growing at a rapid rate, with global manufacturers increasingly realising the benefits of producing parts by AM. According to recent studies, the AM market has observed significant growth, with a projected compound annual growth rate (CAGR) of 18.2% and a valuation of USD 13.89 billion in 2021. It is expected to reach $76.2 billion by 2030, representing a 20.9% CAGR (2022-2030). The AM industry's significant growth trajectory indicates a promising future, driven by its increasing adoption as a mainstream manufacturing process, which offers attractive advantages, including reduced production time and costs, and the ability to create complex designs with superior functionality. An example is newly optimised 3D printed metal aircraft brackets that are 50% lighter, use 90% less material and 90% less energy to produce compared to the equivalent bracket currently produced by machining. Given that AM will continue to grow, future engineers must adapt to this revolutionary manufacturing process, so now is a timely opportunity to introduce a dedicated course that prepares our graduates to work with this technology.}
	{review here}

\course{MATE7016}
	{Materials for Energy Conversion and Storage}
	{4}
	{None}
	{}
	{}
	{Energy storage and conversion materials hold the key to many advanced renewable energy technologies including photo-voltaic systems, lithium-ion and next-generation batteries, hydrogen fuel cells and storage, and superconducting magnetic energy storage. With the increasing need for safe, cost-effective and environmentally friendly methods of energy storage and conversion, it is necessary to accelerate the rate at which energy-related materials are developed. Materials science is an essential enabling technology for emerging renewable technologies. Often, engineering solutions for the energy challenges facing society are constrained by the materials technologies available. This is especially true for energy storage and conversion materials. The aim of this course on Materials for Energy Conversion and Storage is to help future engineers create and develop new materials solutions to meet this global challenge.}
	{review here}

\course{MATH1051}
	{Calculus and Linear Algebra I}
	{4}
	{MATH1050 or a grade of C or higher in Queensland Year 12 Specialist Mathematics (Units 3&4)( or equivalent).}
	{}
	{}
	{Vectors, linear independence, scalar product. Matrices, simultaneous equations, determinants, vector product, eigenvalues, eigenvectors, applications. Equation of straight line and plane. Extreme value theorem, maxima and; minima. Sequences, series, Taylor series, L'Hopital's rules. Techniques of integration, numerical methods, volumes of revolution.}
	{review here}

\course{MATH1052}
	{Multivariate Calculus & Ordinary Differential Equations}
	{4}
	{MATH1050 or a grade of C or higher in Queensland Year 12 Specialist Mathematics (Units 3&4)( or equivalent).}
	{}
	{}
	{Vector calculus, arclength, line integrals, applications. Calculus of 2 & 3 variables: partial derivatives, conservative fields, Taylor series, maxima & minima, non-linear equations. 1st order & linear 2nd order differential equations (constant coefficients). Applications (dynamical systems etc), numerical methods.}
	{review here}

\course{MATH1071}
	{Advanced Calculus & Linear Algebra I}
	{4}
	{A grade of 6 or above in MATH1050; or a grade of B or higher in Queensland Year 12 Specialist Mathematics (Units 3&4)( or equivalent).}
	{}
	{}
	{1. Elementary linear algebra: Vectors, linear independence, scalar product. Matrices, simultaneous equations, determinants, Gaussian elimination, eigenvalues, eigenvectors, applications. Equation of straight line & plane. 2. Introduction to proof-based calculus: Fields, sequences, limits, continuity, intermediate and extreme value theorems, maxima & minima. 3. Techniques of calculus: Series, differentiation, integration, numerical methods, Taylor series, L'Hopital's rule. This course differs from MATH1051 by treating material in more depth and with greater rigour.}
	{review here}

\course{MATH1072}
	{Advanced Multivariate Calculus & Ordinary Differential Equations}
	{4}
	{A grade of 6 or above in MATH1050; or a grade of B or higher in Queensland Year 12 Specialist Mathematics (Units 3&4)( or equivalent).}
	{}
	{}
	{Vector calculus, arc-length, line integrals, applications. Calculus of 2 and; 3 variables: partial derivatives, conservative fields, maxima ; minima. 1st order ; linear 2nd order differential equations (constant coefficients). Applications (dynamical systems etc), numerical methods for non-linear equations and differential equations. Introduction to mathematical modelling and programming. (MATH1072 can be studied concurrently with MATH1051)}
	{review here}

\course{MATH2001}
	{Calculus & Linear Algebra II}
	{4}
	{(MATH1051 or MATH1071) and (MATH1052 or MATH1072)}
	{}
	{}
	{Second order differential equations; undetermined coefficients, variation of parameters. Multi-dimensional calculus; surface & volume integrals, cylindrical, spherical and general coordinate transformations. Stoke's & Green's theorems, applications (flux, heat equations). Linear algebra, diagonalisation, quadratic forms, elementary numerical linear algebra. Taylor series, maxima, minima and saddle points in N-dimensions. Method of least squares for functions. Vector spaces, norms and inner products (for square-integrable functions). Gram-Schmidt orthogonalization and orthogonal matrices.}
	{review here}

\course{MATH2010}
	{Analysis of Ordinary Differential Equations}
	{4}
	{MATH1052 or MATH1072}
	{}
	{}
	{ODE's - Systems variation of constants, fundamental matrix. Laplace transform, transform for systems, transfer function. Stability, asymptotic stability; phase plane analysis.}
	{review here}

\course{MATH3202}
	{Operations Research & Mathematical Planning}
	{4}
	{(MATH1051 or MATH1071) and (MATH1052 or MATH1072)}
	{}
	{}
	{Techniques and applications of optimisation in operations research, including linear programming, integer programming, dynamic programming and meta-heuristics. Use of Python and the Gurobi optimisation package for linear and integer programming.}
	{review here}

\course{MECH2100}
	{Machine Element Design}
	{4}
	{MECH2300 and (MECH2305 or METR2800)}
	{}
	{}
	{Mechanical design principles. Design, manufacture and assembly of basic machine elements. Machine frames, welded, adhesive and bolted joints, fasteners. Stepped shafts and features, rolling element bearings; gear mechanics and manufacture. Design for strength, design for other mechanical failure modes including fatigue, stress concentration. Safety, ergonomics and standards.}
	{review here}

\course{MECH2210}
	{Intermediate Mechanical and Space Dynamics}
	{4}
	{(ENGG1010 or ENGG1400 or ENGG1700) and MATH1051 and MATH1052(ENGG1400 no longer offered)}
	{}
	{}
	{Applications of kinematics and kinetics of particles and rigid bodies; Applications of energy and momentum methods; Vibration of single degree of freedom systems; Balancing of rotating and reciprocating masses; Introduction to orbital mechanics and 3D rigid body dynamics with mechanical and space applications.}
	{review here}

\course{MECH2300}
	{Structures and Materials}
	{4}
	{[ENGG1400 and (ENGG1200 or ENGG1211)] or ENGG1700(ENGG1400, ENGG1200 and ENGG1211 no longer offered)}
	{}
	{}
	{Mechanics of simple structures; Transformation of stress and strain; Linear elasticity; Phase diagrams; Steel microstructures; Strengthening mechanisms of metals; Material failure mechanisms; Corrosion.}
	{review here}

\course{MECH2700}
	{Computational Engineering and Data Analysis}
	{4}
	{(ENGG1001 or CSSE1001) and (MATH1051 or MATH1071)}
	{}
	{}
	{This course provides an introduction to the development and application of numerical methods to resolve common challenges faced in mechanical engineering. It builds an important foundation of knowledge in programming, numerical algorithms, and simulation techniques that is essential for modern day engineers. The course starts with numerical approaches for interpolation, differentiation and quadrature before introducing direct and iterative approaches to solve linear systems. Solution techniques for ordinary differential equations (ODEs) and non-linear equations are formulated and techniques for data fitting and optimisation (inc. least-squares approaches) are developed and applied to engineering problems. The final module of the course formulates computational techniques to solve partial differential equations (parabolic, hyperbolic, and elliptic) with numerous applications involving heat transfer, convection-diffusion, and wave propagation. Completion of this course will provide students with the necessary skills to develop computational approaches to address a wide range of engineering problems.}
	{review here}

\course{MECH3200}
	{Advanced Dynamics and Vibrations}
	{4}
	{MECH2210}
	{}
	{}
	{Discrete (lumped parameter) systems: Lagrange¿s equations, multidegree of freedom systems and applications, vibration isolation and absorption. Continuous (distributed parameter) systems: free and forced vibration, modal analysis, approximate techniques, finite element method. Measurements and Applications: vibration measurement techniques, vibration control.}
	{review here}

\course{MECH3250}
	{Engineering Acoustics}
	{4}
	{MATH1051 and MATH1052 and (ENGG1400 or ENGG1700)(ENGG1400 no longer offered)}
	{}
	{}
	{Plane sound waves; physical aspects of sound; the human ear; physiological aspects of sound; sound level meters; statistical noise measures; occupational noise; road-traffic noise; directivity of sound; reflection & transmission of sound; sound in enclosed spaces; engineering acoustics applications.}
	{review here}

\course{MECH3301}
	{Materials Selection}
	{4}
	{MATE1000, ENGG1200, ENGG1211 or ENGG1700.(MATE1000, ENGG1200 and ENGG1211 no longer offered)}
	{}
	{}
	{This course explores the principles and practices of material selection for mechanical design, covering structural requirements, shape considerations, as well as economic and environmental impacts. Students will utilise material databases, material indices, and selection charts to make informed decisions. Through a practical project, students will gain hands-on experience in material selection tailored for real-world applications.}
	{review here}

\course{MECH3780}
	{Computational Mechanics}
	{4}
	{MECH2410 and MECH2700 and MECH2300}
	{}
	{}
	{Computational analysis of problems in solid and fluid mechanics using the finite element (FEM) and finite volume (FVM) methods, respectively, in the context of product lifecycle management (PLM). The FEM component addresses strain energy, interpolation functions, element types, two-dimensional analysis types, plate and shell models, and an introduction to nonlinear analysis. The computational fluid dynamics (i.e. FVM) component addresses transient and steady-state flows, laminar and turbulent regimes, boundary layer and turbulence modelling, and stability criteria. Topics common to both approaches include problem decomposition, appropriate selection of boundary conditions, grid convergence, model verification and validation, and data visualisation and interrogation.}
	{review here}

\course{MECH4304}
	{Net Shape Manufacturing}
	{4}
	{MECH2305}
	{}
	{}
	{Net-shape manufacturing of metals & ceramics processes: casting from liquid state & consolidation of components from powders pressed into almost finished complex shapes. Understanding of the principles of solidification & powder processing & principles used in the manufacture of components.}
	{review here}

\course{MECH4950}
	{Advanced Manufacturing in Practice}
	{4}
	{Permission of Head of School}
	{}
	{}
	{(Course is offered on an occasional basis.) Topics and content to be determined by student interest and availability of visiting staff. For details, consult course coordinator. For information about how to enrol in this course, please email studentenquiries@mechmining.uq.edu.au.}
	{review here}

\course{MECH7101}
	{Design of Experiments}
	{4}
	{ENGG7601(None now because ENGG7601 is no longer offered)}
	{}
	{}
	{Students will learn how to design experiments to explore the entire parameter space for an engineering problem; they will learn how to test hypotheses to a desired degree of confidence; they will learn how to process data from engineering sensors and how to analyse such data using advanced multivariate statistics.}
	{review here}

\course{METR2800}
	{Mechatronic System Design Project 1}
	{4}
	{ENGG1100}
	{}
	{}
	{Introduction to mechatronic engineering. Technical: mechatronic technology exemplars; mechanical & electrical drawing; small mechatronic product designed & tested for potential client. Organisational: project team must follow standard procedures - milestones, reporting, project meetings, interacting with client.}
	{review here}

\course{METR3100}
	{Control System Implementation}
	{4}
	{ENGG1300}
	{}
	{}
	{METR3100 introduces students to engineering frameworks that support the design and implementation of safe, robust control systems. Students are shown how to identify and mitigate against hazards using a system theoretic approach to managing risk. The course also explores the operating principles of sensing, logic, and actuation subsystems that comprise an overall control system.}
	{review here}

\course{METR4201}
	{Control Engineering 1}
	{4}
	{MECH2210 or ELEC2004}
	{}
	{}
	{Introduction to control system design; system modelling principles for electrical & mechanical systems; the Laplace transform; block diagram modelling; open & closed loop control; role of feedback; transient & steady state performance; root locus; frequency response analysis; compensator design, practical issues in the implementation of control systems. In 2019, the course has been redesigned as part of the UQ2U program. This approach is intended to provide students with a signature UQ experience that maximises campus-based 'face time' with academics and peers, in combination with high-value, flexible online learning.}
	{review here}

\course{METR4202}
	{Robotics and Automation}
	{4}
	{ELEC3004 or METR3200 or METR4201(METR3200 is no longer offered)}
	{}
	{}
	{Modern robotic algorithms and techniques for use in practical applications. Coverage of advanced navigation, motion planning and control methodologies for intelligent robotic systems.}
	{review here}

\course{METR4810}
	{Mechatronic System Design Project II}
	{4}
	{METR2800}
	{}
	{}
	{Technical: Small teams of students undertake design, implementation, testing, evaluation and presentation of mechatronic systems of intermediate size and complexity. Organisation: project team must follow standard procedures, milestones, reporting, project meetings, interacting with client. }
	{review here}

\course{METR4911}
	{Thesis/Design Project}
	{4}
	{None}
	{}
	{}
	{(May commence Sem 2) Thesis/design project on topic in mechatronic engineering. Students commencing in Semester 1 enrol in METR4911 for Semester 1 (Part A) and Semester 2 (Part B); students commencing in Semester 2 enrol in METR4912 for Semester 2 (Part A) and the following Semester 1 (Part B).}
	{review here}

\course{METR4912}
	{Thesis/Design Project}
	{4}
	{None}
	{}
	{}
	{(May commence Sem 2) Thesis/design project on topic in mechatronic engineering. Students commencing in Semester 1 enrol in METR4911 for Semester 1 (Part A) and Semester 2 (Part B); students commencing in Semester 2 enrol in METR4912 for Semester 2 (Part A) and the following Semester 1 (Part B).}
	{review here}

\course{METR6203}
	{Control Engineering 2}
	{4}
	{METR4201 or METR7200(METR7200 no longer offered)}
	{}
	{}
	{Coverage of various advanced topics in control systems engineering: (i) observers and state estimation, (ii) multivariable systems in the frequency domain, (iii) robust control, and (iv) model predictive control.}
	{review here}

\course{MINE3110}
	{Integrated Orebody Knowledge}
	{4}
	{None}
	{}
	{}
	{This course will introduce the concepts of resource and reserve estimation in both coal and metalliferous deposits. This represents a critical component in the life cycle of a mine and provides the link between the processes of exploration geology and mine planning.}
	{review here}

\course{MINE3122}
	{Mining Systems and Automation}
	{4}
	{None}
	{}
	{}
	{This course presents a systems approach to the principles, design and application of the major surface and underground mining methods together with the associated equipment, services and infrastructure. Furthermore, the course provides an introductory overview of automation and its diverse applications in both surface and underground mining contexts.}
	{review here}

\course{MINE3123}
	{Mine Planning and Sustainability}
	{4}
	{WILL NO LONGER BE OFFERED(read description)}
	{}
	{}
	{ None: This course will be discontinued from 2025. Students should enrol into MINE4127.}
	{review here}

\course{MINE3129}
	{Applied Mining Geomechanics}
	{4}
	{WILL NO LONGER BE OFFERED(read description)}
	{}
	{}
	{This course will be discontinued from 2025. Students should enrol into MINE6112.}
	{review here}

\course{MINE4124}
	{Mine Design and Feasibility}
	{4}
	{None}
	{}
	{}
	{Development of a pre-feasibility study for a metalliferous or coal mining project. Activities include assessment of reserves, method selection, layout and optimisation of surface and underground operations, geotechnical design, ventilation design, project risk assessment, mine scheduling, equipment selection, cost estimation, economics / finance, and sustainability. Mine design and optimisation software packages are used.}
	{review here}

\course{MINE4127}
	{Mine Planning and Sustainability}
	{4}
	{MINE3122}
	{}
	{}
	{Theoretical principles and practical methodologies associated with mine planning. Mine Planning is an iterative process entailing elements of design, scheduling, evaluation, and the incorporation of Environmental, Social and Governance (ESG) risks and opportunities across the value chain. Aspects of long term strategic mine planning as well as short term tactical planning will be investigated. Some recognised industry software products are used.}
	{review here}

\course{MINE4129}
	{Mine Process Optimisation}
	{4}
	{MINE3122}
	{}
	{}
	{In the past, ore fragmentation, comminution, processing, and concentration were viewed as isolated activities, each optimised separately. Today, the approach has evolved significantly, with every process integrated into the overall recovery effort. This course introduces key concepts of the mineral extraction value chain, placing a strong emphasis on the initial comminution stage (i.e. drilling and blasting) and its influence on subsequent downstream activities. Additionally, the course will introduce practical tools and processes used to implement integrated optimisation strategies, such as ¿mine to mill' and value-based ore control.}
	{review here}

\course{MINE6112}
	{Applied Mining Geomechanics}
	{4}
	{MECH2300 or CIVL2210}
	{}
	{}
	{Understanding of geomechanical behaviour of orebody and host rock is of critical importance to achieve safe and efficient mining operations and has become an integral part of a mine design and planning. As mines are getting deeper, the ground in situ stress and water pressure are also increasing, which requires more in-depth understanding of rock mass to address this emerging grant challenge in deep mining. This course provides students with a fundamental knowledge of rockmass properties, and with a practical understanding of the applications of geotechnical engineering principles in mining from the perspective of planning, design, and operations. This will enable students to understand and apply fundamental concepts and design methodologies to design safe excavations in both surface and underground mines.}
	{review here}

\course{STAT2003}
	{Mathematical Probability}
	{4}
	{MATH1051 or MATH1071}
	{}
	{}
	{Probability; random variables; probability distributions; Markov processes; statistical analysis & modelling}
	{review here}

\course{STAT2004}
	{Statistical Modelling & Analysis}
	{4}
	{((MATH1051 or MATH1071)+(STAT1201, STAT1301 or STAT2201)+(STAT2003 or STAT2203))}
	{}
	{}
	{Statistical inference; parametric models; point estimation; properties of estimators; maximum likelihood (ML) and properties of ML estimators; confidence intervals; hypothesis testing; goodness-of-fit tests; Bayesian inference; ANOVA; linear and logistic regression.}
	{review here}

\course{STAT2201}
	{Analysis of Engineering & Scientific Data}
	{4}
	{MATH1050 or ;a grade of C or higher in Queensland Year 12 Specialist Mathematics(Units3&4)( or equivalent).}
	{}
	{}
	{Statistical models & analyses required for analysing engineering & scientific data, including sampling methods, exploratory data analysis, standard probability models, estimation, hypothesis tests, regression, experimental design.}
	{review here}

\course{TIMS3309}
	{Technology and Innovation Management}
	{4}
	{None}
	{}
	{}
	{Introduction to the management of technology and innovation, including strategic and operational technology and innovation management, business competitiveness, business partnerships and alliances, managing R&D, new product development, and valuation of technology. The course is intended to develop corporate entrepreneurs who effectively manage innovation leading firms.}
	{review here}

